%
%
%
\section{Closing remarks \& Recommended reading}
\label{chap:further_reading}

At this point, the reader should have a conceptual grasp of the essentials of a plane-wave density functional theory code. With this knowledge, the reader should be comfortable studying a \textit{simple} and \textit{well-documented} plane-wave density functional theory code. For this very purpose, I have written such a code, including documentation and examples, which can be found at \url{https://pypwdft.imc-tue.nl}. 

To further practice with the material in this document, I have also prepared another repository that contains a set of exercises. These exercises can be found at \url{https://github.com/ifilot/hsl-pwdft-exercises}.

Last but not least, the list below provides a set of useful text books for the reader to further practice their understanding of density functional theory or to brush up on the fundamental knowledge of electronic structure calculations.

\begin{itemize}
    \item For the absolute beginner, it is recommended to read a introduction textbook on quantum mechanics. For a somewhat more fundamental introduction, \textit{Introduction to Quantum Mechanics}\cite{2004:griffiths} by David J. Griffiths is recommended. For a more chemistry flavored perspective, \textit{Molecular Quantum Mechanics}\cite{2011:atkins} by Peter W. Atkins is a nice alternative.
    
    \item Once the basics of quantum mechanics is mastered, one is strongly recommended to follow-up with the seminal text book \textit{Modern Quantum Chemistry: Introduction to Advanced Electronic Structure Theory}\cite{szabo} by Attila Szabo and Neil S. Ostlund. Chapters 2 and 3 are excellent in fundamentally understanding electronic structure calculations. Chapter 1 offers a solid mathematical recap. Chapter 4 and further delve into post-Hartree-Fock calculations, which pending on the field of research of the reader might be valuable.

    \item For an in-depth exploration of density functional theory, the book \textit{Density-Functional Theory of Atoms and Molecules}\cite{1994:parr} provides a detailed overview.

    \item An overview on build a simple Hartree-Fock or localized-orbital density functional theory code from scratch, one can read the book \textit{Elements of Electronic Structure Theory} from this author. A free copy of this book can be obtained via \url{https://ifilot.pages.tue.nl/elements-of-electronic-structure-theory}.
    
\end{itemize}