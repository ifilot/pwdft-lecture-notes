%
%
%
\section{Introduction}

%
%
%
\subsection{Purpose and scope}

This document serves as supplementary background material for the  course ``Plane Wave Density Functional Theory from Scratch'' as part of the 2024 iteration of the Han-sur-Lesse Winterschool. The author of this document assumes that the reader has an academic background in quantum mechanics, theoretical and computational chemistry, calculus and linear algebra. In \cref{chap:further_reading} a number of useful educational sources are provided to assist the reader. Based on the experience of the author and his perception of the various educational curricula, certain aspects are explained in more detail whereas others are only glossed over or omitted altogether.

The digital version of this document is annotated with hyperlinks throughout the text. Thus, if one uses the digital version and clicks on any of the references, one is automatically referred to the corresponding Equation, Figure, Table, or note.

%
%
%
\subsection{Notation used in this work}

In this work, we will be dealing with scalars, vectors and matrices. To differentiate between these three, matrices are always written using uppercase characters in bold (e.g. $\mathbf{M}$), vectors are indicated by lowercase characters using an arrow (e.g. $\vec{v}$) and scalars have neither bold text nor arrows (e.g. $c$). Sets are indicated by curly braces, for example, the set of numbers 1-4 is indicated by $\{ 1,2,3,4 \}$.

%
%
%
\subsection{Units}
Throughout this text, atomic units are being used.\cite{szabo} In these units, many equations simplify dramatically. All distances are provided in Bohr units, equivalent to $5.29 \times 10^{-11}$ m. Energies are given in Hartrees (Ht), with one Hartree approximately corresponding to $4.36 \times 10^{-18}$ or $27.21$ eV.

%
%
%
\subsection{Source code}
Two Python-based repositories are associated with this document. One containing a fully working plane-wave density functional theory code, including detailed working documentation. Details can be found via \url{https://pypwdft.imc-tue.nl}. Another repository contains a set of exercises to allow the reader to assess their mastering of the lecture material. These exercises can be found at \url{https://github.com/ifilot/hsl-pwdft-exercises}.

%
%
%
\subsection{Feedback, comments and questions}

Although the aim for this document is to be sufficiently complete, clear and free of errors, the practicality of life and the finite (limited) brain capacity of the author typically prevents this from being the case. To nevertheless optimize the learning experience of the reader, the author welcomes and appreciates feedback, comments and questions which can be provided in-person or alternatively by digital means via i.a.w.filot@tue.nl.